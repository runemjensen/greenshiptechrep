%1)	Problem is important
%a.	Program analysis, done
%b.	Constraint logic programming, done
%c. - logic programming
%d. - do-loops
\section{Introduction}
Systems of linear constraints are used to model dependencies among variables in a broad variety of domains. 
Fourier-Motzkin-elimination (FME) is a projection method, which removes a variable from such a system such that the feasible values and inter-dependencies of the remaining variables in the resulting system are unaltered. In this way, FME computes an abstraction of the original system, where the value of the eliminated variable no longer is represented.

%%''Sets of linear inequalities are an expressive reasoning tool for approximating the reachable states of a %program'' \cite{simon10}
%In program analysis, polyhedra are widely used as abstract domain for expressing information about numerical variables \cite{cousot78}. In sequential program analysis, for instance, linear constraints can capture information about the possible values and relationship of values of the numerical variables after each instruction in a program (\red{e.g. \cite{fouilhe}}). The value domain of a local variable, however, does not need to be represented after it goes out of scope and can be removed with a projection algorithm such as FME (if the polyhedron is represented as a linear inequality system). 

Convex polyhedra\footnote{Convex polyhedra can be described either as a conjunction of halfspaces given by constraints, or by a set of generators (points and rays).} are widely used as an abstract domain in static analysis of programs \cite{cousot78}.
In sequential program analysis, for instance, convex polyhedra can capture information about the possible values and relationship of values of the numerical variables after each instruction in a program ({e.g. \cite{fouilhe}}). The value domain of a local variable, however, does not need to be represented after it goes out of scope and can be removed with a projection algorithm such as FME, if the polyhedra are represented as linear inequality systems. 

%Further, ''Sets of linear inequalities are an expressive reasoning tool for approximating the reachable states of a program'' \cite{simon10}
Polyhedra are also widely used in analysis of logic programs and constraint logic programs (see e.g. applications and references in \cite{benoy05}). %, since they can model the reachable states of a program \cite{simon10}. 
Here projection plays a similar role %(e.g. \cite{benoy05}, \cite{jaffar93}) 
of eliminating auxiliary variables introduced during the execution of a program. Similarly, projection of polyhedra is used in constraint query languages, %\cite{kanellakis90}[=Ref 15 Imb93], 
where a representation of the set of answers to a query is requested that is only expressed in the query-variables and hence auxilliary variables need to be projected (e.g. \cite{lassez90}).  
%Linear constraints (on numerical values) can also be used to represent/model/analyse e.g. reachability and safety properties, as well as many other relevant properties; see \cite{} for examples and references.

Other applications of FME include coordination and/or negotiation situations as suggested in \cite{lukatskii08}, 
%where different parties have each their interest influenced by some common factors, 
%\red{and elimination of irrelevant variables in biophysics models \cite{shapot12}}, 
or as an alternative solution method for parametric linear programming, since problems of this type can be formulated as a projection of an inequality system \cite{jones08}.
\\
\\
%1)	Our application
%a.	Stowage optimization models
%b.	Compute stowage capacity models
% -c. The current ones are not good enough
The motivation for the decomposed FME algorithm presented in this report is to compute capacity models of mega container vessels. Capacity models are important in practice for liner shipping companies to optimize decisions about uptake management on services and network flow. A capacity model says how many containers of each type and weight class, the vessel can carry, and in particular it {quantifies} the dependencies between these numbers of containers. %/how the number of each type depends on and/or influences each other. 
%describe the dependencies between the capacities of various type of containers  

However, the capacity models currently considered e.g. within models for revenue management ({e.g., \cite{ting04}, \cite{feng08}, \cite{zurheide13}}) only consists of upper bounds on a few parameters and hence ignore several important aspects of stowage, resulting in considerable overestimation of the capacity \cite{AlbertosThesis}.


However, capacity models can be derived via projection from a much more detailed stowage model that {additionally} represents where on the vessel, the containers are stowed (e.g. {\cite{pacino11}, \cite{pacino12} and \cite{AlbertosThesis}}). In this way, a capacity model is an abstraction of a stowage model, where variables representing location information are removed using FME.  
%
%In this report, we present a method for obtaining \red{reasonable sized} yet accurate capacity models from stowage models.
\\\\
%%%%%%%%%%%%%%%%%%%%%%%%%%%%%%%%%%%%%%%%%%%%%%%%%%
%1)	Scalable FME algorithms
%a.	Basic elements
%2)	Our unique contribution: block structured decomposition
%3)	Key 
In this report, we introduce a novel hierarchically decomposed and parallelized FME algorithm for massive variable elimination. Similar to previous FME frameworks (e.g., \cite{simon05}, \cite{lukatskii08}, \cite{shapot12}), it applies 
%preprocessing (including removal of linearly dependent inequalities), Gauss-elimination, Fourier-Motzkin-elimination, parallel, full redundancy removal and \red{adjustment} of the edges of the projecting (i.e. an approximation) using sequential redundancy removal of ``almost redundant'' inequalities. Further, the method uses a \red{novel} decomposition of the input problem to exploit its block structure. 
{preprocessing}, Gauss\--eli\-mi\-na\-tion (equality removal), removal of syntactic and quasi-syntactic redundancies, complete removal of redundant inequalities between each variable elimination together with an approximation of the boundaries of the projection. Our main contribution is a hierarchical decomposition of the problem using auxiliary variables based on the frequent block structure of linear programs as well as a sound parallelization of the redundancy removal. Further, we aim at simplifying the resulting system by reducing the number of variables by two orders of magnitude.

Theoretically the number of inequalities may grow double exponentially with the number of eliminated variables. Our experiments on deriving capacity models from stowage models of mega vessels, however, show only a modest growth in the number of inequalities  when complete removal of redundant inequalities between each variable elimination is carried out (e.g., see Figures~\ref{fig:FinalProjectionS1}-\ref{fig:FinalProjectionS3} in Chapter~\ref{sec:results}). The number of inequalities even decrease in the end where most variables are removed. The resulting capacity models are small and useful in practice (e.g., see Table~\ref{tab:results} in Chapter~\ref{sec:results}). This is a surprising finding, but since we only exploit the block structure of stowage models, we believe that similar results can be obtained in other domains.
\\
\\
The remainder of this report is organized as follows.
In Chapter~\ref{sec:def} we present the definitions and notation relating to inequality systems and projections that are used in this report. 
Then we proceed in Chapter~\ref{sec:basic} by describing the basic algorithms used for achieving the projection, namely the classical Fourier-Motzkin elimination, Gauss-elimination, and some basic methods for preprocessing and removing redundant information (inequalities). 
In Chapter~\ref{sec:improvements} we then describe the alterations and improvements made to these algorithms. Specifically we detail an altered and parallellized method for removing redundant inequalities, as well as how our system (as well as other block structured systems) can be decomposed to achieve a more efficient projection.    
Chapter~\ref{sec:implementation} then gives an overview on how the different parts described so far are tied together to achieve the wanted goal, and a few details regarding the implementation of the procedure are given as well.
Subsequently, Chapter~\ref{sec:results} gives a short presentation of the considered stowage model(s) after which the results of projecting these are presented. 
Finally, we review and discuss related work in Chapter~\ref{sec:related} before Chapter~\ref{sec:conclusion} concludes.
{The appendix contains proofs of lemmas and propositions stated in the text.}